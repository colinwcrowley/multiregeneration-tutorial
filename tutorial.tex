\documentclass[12pt]{article}
\usepackage[margin=1.1in]{geometry}
\usepackage{enumerate,amsmath,amsthm,amssymb,amsfonts,mathtools}
\usepackage{xcolor}
\usepackage{listings}
\title{\large \bf
Title
}
\author{}
\newcommand{\norm}[1]{\left\lVert#1\right\rVert}                                                                                                      
\newcommand{\R}{\mathbb{R}}
\newcommand{\Div}{\text{Div}}
\newcommand{\Hom}{\text{Hom}}
\newcommand{\Jac}{\text{Jac}}
\newcommand{\rk}{\text{rk}\ }
\newcommand{\coker}{\text{coker}\ }
\newcommand{\Princ}{\text{Princ}}
\newcommand{\Q}{\mathbb{Q}}
\newcommand{\C}{\mathbb{C}}
\newcommand{\Z}{\mathbb{Z}}
\newcommand{\F}{\mathbb{F}}
\newcommand{\V}[1]{\mathbf{#1}}
\newcommand{\Spt}{\text{support}}
\begin{document}
\maketitle
\section{Getting Started}
Say that we are given $n$ polynomials
with complex coefficient in $n$ variables, such as the 
following. 
\begin{align*}
    f_1 &= (x-1)(y-3)\\
    f_2 &= (x-2)(y-4)\\
\end{align*}
Our goal is to describe algorithmically all points of the set 

\[
    \mathcal{V}(f_1, f_2) = \{(z_1,z_2) \in \C^2 : f_1(z_1, z_2) = 
    f_2(z_1, z_2) = 0\}
\]
which we call the \emph{zero locus} or the \emph{vanishing} of $f_1$ and 
$f_2$. 

B\'{e}zout's Theorem (one formulation of it) states that if 
$f_1, \ldots, f_n$ are complex polynomials in $n$ variables of degrees 
$d_1, \ldots, d_n$ and $\mathcal{V}(f_1, \ldots, f_n)$ is finite, then 
the size of $\mathcal{V}(f_1, \ldots, f_n)$ is at most $d_1d_2 \ldots d_n$. 
For now we will assume that our system has finitely many solutions.
Therefore the degrees $d_1, \ldots, d_n$ give an upper bound on the size 
of the output, and in fact they are necessary for the algorithm itsself.

To solve the system above, first we declare the names of our variables 
in a file named ``bertiniInput\_variables'' with the following contents.

\begin{lstlisting}
    variable_group x,y; 
\end{lstlisting}

Next we declare the equations in a file named ``bertiniInput\_equations'' 
with the following contents.

\begin{lstlisting}
    function f1,f2;
    f1 = (x-1)*(y-3);
    f2 = (x-2)*(y-4);
\end{lstlisting}

Create a file called ``bertiniInput\_trackingOptions'' which we will 
leave blank for now.

Finally, create a file called ``inputFile.py'' with the following 
contents
\begin{lstlisting}
degrees = [[2], [2]]
workingDirectory = "run"
\end{lstlisting}

Now we 

\section{Multiple variable groups}
\section{Projective variable groups}
\section{Nonsquare systems}




\end{document}
